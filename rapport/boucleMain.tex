Le main : Le jeu va se passer dans le fichier Class "Main".
On commence par afficher les règles en appelant la méthode static de la classe Rules
afficherRegles.
On crée ensuite une nouvelle instance gameAvecUnPetitG  de type Game.
Puis on affiche les différents types de jeu disponible avec la méthode static ruleChoice
de la classe Translation.
On crée un entier choice dont l'utilisateur doit choisir la valeur selon le jeu qu'il a choisi.
On entre dans la boucle while et on en sort que si choice <0 ou que choice > à l'attribut nbRules
de la classe Rules.
Dans cette boucle, on appelle la méthode newGame() de Game avec l'instance qu'on a crée pour créer une nouvelle partie, et on lui met en paramètre le choix de règles qu'on a choisies précédemment.
On lance ensuite la boucle du jeu avec loopGame() de Game, puis quand le jeu est fini, c'est la méthode endGame() qui est appelée.
Finalement on affiche de nouveau les règles et l'utilisateur choisit si il veut refaire une partie ou quitter la boucle en appuyant sur 0.